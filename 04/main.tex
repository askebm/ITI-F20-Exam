\documentclass[aspectratio=169,10pt,t]{beamer}
% \usetheme[
% %%% options passed to the outer theme
% %    progressstyle=fixedCircCnt,   %either fixedCircCnt, movCircCnt, or corner
% %    rotationcw,          % change the rotation direction from counter-clockwise to clockwise
% %    shownavsym          % show the navigation symbols
%   ]{SDUsimple}
\usepackage{SDUtheme/beamerthemeSDUsimple}
% If you want to change the colors of the various elements in the theme, edit and uncomment the following lines
% Change the bar and sidebar colors:
%\setbeamercolor{SDUsimple}{fg=red!20,bg=red}
%\setbeamercolor{sidebar}{bg=red!20}
% Change the color of the structural elements:
%\setbeamercolor{structure}{fg=red}
% Change the frame title text color:
%\setbeamercolor{frametitle}{fg=blue}
% Change the normal text color background:
%\setbeamercolor{normal text}{fg=black,bg=gray!10}
% ... and you can of course change a lot more - see the beamer user manual.
\usepackage{color}
\usepackage{float}
\usepackage{dsfont}                         % Enables double stroke fonts
\usepackage{bm}
\usepackage[utf8]{inputenc}
\usepackage[english]{babel}
\usepackage[T1]{fontenc}
\usepackage{booktabs}
\usepackage{minted}
\setminted{linenos=true,breaklines=true,obeytabs=true,tabsize=2}
\usepackage{import}
\newcommand{\includesvg}[2][./]{%
    \immediate\write18{inkscape #1#2.svg --export-latex -D --export-filename=#1#2.pdf }
\import{#1}{#2.pdf_tex}
}

\usepackage{pgf}
\usetikzlibrary{graphs,graphdrawing,quotes}
\usegdlibrary{force,layered,circular}

% Or whatever. Note that the encoding and the font should match. If T1
% does not look nice, try deleting the line with the fontenc.
\usepackage{helvet}
\usefonttheme{professionalfonts}

\newtheorem{algorithm}{Algorithm}
%\newtheorem{problem}{Problem}
\newtheorem{proposition}{Proposition}
% colored hyperlinks
\newcommand{\chref}[2]{%
  \href{#1}{{\usebeamercolor[bg]{SDUsimple}#2}}%
}

\title{MySQL and PHP}
\subtitle{Hardcore backend}
%\date{\today}
\date{ }

\author{
  \textbf{ITI-F20}
}


% - Give the names in the same order as they appear in the paper.
% - Use the \inst{?} command only if the authors have different
%   affiliation. See the beamer manual for an example

\institute[
%  {\includegraphics[scale=0.2]{SDU_segl}}\\ %insert a company, department or university logo
  SDU Robotics\\
  The Maersk Mc-Kinney Moller Institute\\
  University of Southern Denmark
] % optional - is placed in the bottom of the sidebar on every slide
{% is placed on the bottom of the title page
  SDU Robotics\\
  The Maersk Mc-Kinney Moller Institute\\
  University of Southern Denmark

  %there must be an empty line above this line - otherwise some unwanted space is added between the university and the country (I do not know why;( )
}

% specify a logo on the titlepage (you can specify additional logos an include them in
% institute command below
\pgfdeclareimage[height=0.5cm]{titlepagelogo}{SDUgraphics/SDU_logo_new} % placed on the title page
%\pgfdeclareimage[height=1.5cm]{titlepagelogo2}{SDUgraphics/SDU_logo_new} % placed on the title page
\titlegraphic{% is placed on the bottom of the title page
  \pgfuseimage{titlepagelogo}
%  \hspace{1cm}\pgfuseimage{titlepagelogo2}
}

\begin{document}
% the titlepage
{\SDUwavesbg%
\begin{frame}[plain,noframenumbering] % the plain option removes the header from the title page
  \titlepage
\end{frame}}
%%%%%%%%%%%%%%%%

% TOC
\begin{frame}{Agenda}{\vphantom{(y}}
	\vfill
	\begin{itemize}
		\item Structure
		\item Syntax
		\item PDO
	\end{itemize}
	\vfill
\end{frame}
%%%%%%%%%%%%%%%


\begin{frame}[fragile]
	\frametitle{Structure}
	\framesubtitle{Database or Schema, who knows?}
	\vfill
	\begin{figure}[h]
	\begin{center}
	\begin{tikzpicture}[scale=1, transform shape]
		\graph[layered layout]{
			"DBMS"[draw] // {
				Schema1 [draw] // {
					table1[draw,minimum width=5cm,minimum height=3cm]
				},
				Schema2 [draw]// {
					table2[draw,minimum width=5cm,minimum height=3cm]
				}
			}
		};
	\end{tikzpicture}
	\end{center}
	\end{figure}
	\vfill
\end{frame}

\begin{frame}[fragile]
	\frametitle{MySQL Syntax}
	\framesubtitle{EVERYTHINGS IS IN CAPS!}
	\begin{minted}{mysql}
DROP DATABASE IF EXISTS dbName;
CREATE DATABASE dbName;
USE dbName;

CREATE TABLE user (
	user_id INT UNSIGNED PRIMARY KEY AUTO_INCREMENT,
	username VARCHAR(100) NOT NULL,
	password VARCHAR(100) NOT NULL
);
	\end{minted}
	\pause
	\vfill
	\begin{minted}{mysql}
SELECT username FROM user WHERE user_id=1;
	\end{minted}
\end{frame}

\begin{frame}[fragile]
	\frametitle{PDO}
	\framesubtitle{One statement to rule them all}
	\begin{onlyenv}<1>

		\begin{minted}{php}
<?php
try {
	require 'config.php';
	$conn = new PDO(
		"mysql:host=$servername;dbname=$dbname",
		$username,
		$password,
		array(PDO::ATTR_ERRMODE => PDO::ERRMODE_EXCEPTION)
	);
} catch (PDOException $e) {
	echo "Error: " . $e->getMessage();
}
		\end{minted}
	\end{onlyenv}
	\begin{onlyenv}<2>
		\begin{minted}{php}
<?php
$stmt = $conn->prepare("SELECT * FROM user WHERE username = :username");
$stmt->bindParam(':username',$sane_user);
$stmt->execute();
$result = $stmt->fetchAll();
		\end{minted}
	\end{onlyenv}
\end{frame}

\end{document}
