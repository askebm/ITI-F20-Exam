\documentclass[aspectratio=169,10pt,t]{beamer}
% \usetheme[
% %%% options passed to the outer theme
% %    progressstyle=fixedCircCnt,   %either fixedCircCnt, movCircCnt, or corner
% %    rotationcw,          % change the rotation direction from counter-clockwise to clockwise
% %    shownavsym          % show the navigation symbols
%   ]{SDUsimple}
\usepackage{SDUtheme/beamerthemeSDUsimple}
% If you want to change the colors of the various elements in the theme, edit and uncomment the following lines
% Change the bar and sidebar colors:
%\setbeamercolor{SDUsimple}{fg=red!20,bg=red}
%\setbeamercolor{sidebar}{bg=red!20}
% Change the color of the structural elements:
%\setbeamercolor{structure}{fg=red}
% Change the frame title text color:
%\setbeamercolor{frametitle}{fg=blue}
% Change the normal text color background:
%\setbeamercolor{normal text}{fg=black,bg=gray!10}
% ... and you can of course change a lot more - see the beamer user manual.
\usepackage{color}
\usepackage{float}
\usepackage{dsfont}                         % Enables double stroke fonts
\usepackage{bm}
\usepackage[utf8]{inputenc}
\usepackage[english]{babel}
\usepackage[T1]{fontenc}
\usepackage{booktabs}
\usepackage{minted}
\setminted{linenos=true,breaklines=true,obeytabs=true,tabsize=2}
% Or whatever. Note that the encoding and the font should match. If T1
% does not look nice, try deleting the line with the fontenc.
\usepackage{helvet}
\usefonttheme{professionalfonts}

\newtheorem{algorithm}{Algorithm}
%\newtheorem{problem}{Problem}
\newtheorem{proposition}{Proposition}
% colored hyperlinks
\newcommand{\chref}[2]{%
  \href{#1}{{\usebeamercolor[bg]{SDUsimple}#2}}%
}

\title{JavaScript and RegEx}
\subtitle{Do stuff in browsers}
%\date{\today}
\date{ }

\author{
  \textbf{ITI-F20}
}

% - Give the names in the same order as they appear in the paper.
% - Use the \inst{?} command only if the authors have different
%   affiliation. See the beamer manual for an example

\institute[
%  {\includegraphics[scale=0.2]{SDU_segl}}\\ %insert a company, department or university logo
  SDU Robotics\\
  The Maersk Mc-Kinney Moller Institute\\
  University of Southern Denmark
] % optional - is placed in the bottom of the sidebar on every slide
{% is placed on the bottom of the title page
  SDU Robotics\\
  The Maersk Mc-Kinney Moller Institute\\
  University of Southern Denmark

  %there must be an empty line above this line - otherwise some unwanted space is added between the university and the country (I do not know why;( )
}

% specify a logo on the titlepage (you can specify additional logos an include them in
% institute command below
\pgfdeclareimage[height=0.5cm]{titlepagelogo}{SDUgraphics/SDU_logo_new} % placed on the title page
%\pgfdeclareimage[height=1.5cm]{titlepagelogo2}{SDUgraphics/SDU_logo_new} % placed on the title page
\titlegraphic{% is placed on the bottom of the title page
  \pgfuseimage{titlepagelogo}
%  \hspace{1cm}\pgfuseimage{titlepagelogo2}
}

\begin{document}
% the titlepage
{\SDUwavesbg%
\begin{frame}[plain,noframenumbering] % the plain option removes the header from the title page
  \titlepage
\end{frame}}
%%%%%%%%%%%%%%%%

% TOC
\begin{frame}{Agenda}{\vphantom{(y}}
	\vfill
	\begin{itemize}
		\item JavaScript
		\item Ajax
		\item RegEx
	\end{itemize}
	\vfill
\end{frame}
%%%%%%%%%%%%%%%


\begin{frame}[fragile]
	\frametitle{JavaScript}
	\framesubtitle{Games and stuff}

	Responsive webpage

	\begin{itemize}
		\item<1-> Changing elements\\
			\mintinline{js}{document.getElementById("myBtn").style.color="black";} 
		\item<2-> Add event listeners in HTML\\
			\mintinline{html}{<input type="text" id="fname" onclick="myFunction()">} 
		\item<3-> Directly in JavaScript\\
			\mintinline{js}{document.getElementById("myBtn").addEventListener("click", displayDate);} 
	\end{itemize}

\end{frame}

\begin{frame}[fragile]
	\frametitle{JavaScript}
	\framesubtitle{AJAX}
	\begin{minted}{js}
function loadDoc() {
var xhttp = new XMLHttpRequest();
xhttp.onreadystatechange = function() {
	if (this.readyState == 4 && this.status == 200) {
	 document.getElementById("demo").innerHTML = this.responseText;
	}
};
xhttp.open("GET", "ajax_info.txt", true);
xhttp.send();
}
	\end{minted}
	\footnote{ Example from \href{https://www.w3schools.com/js/js_ajax_intro.asp}{w3schools}}

\end{frame}


\begin{frame}[fragile]
	\frametitle{JavaScript}
	\framesubtitle{RegEx}
	\begin{itemize}
		\item<1-> Phone Number Validation\\
			\mintinline{js}|let phoneRegEx = new RegExp(/^\+(\d){8,25}$/); | 
		\item<2-> Password validation\\
			\mintinline{js}{let ref = new RegExp(/^(?=.*[a-z])(?=.*[A-Z])(?=.*\d)(?=.*[@$!%*?&])[A-Za-z\d@$!%*?&]{8,}$/) } 
	\end{itemize}
\end{frame}

\begin{frame}[fragile]
	\frametitle{PHP}
	\framesubtitle{RegEx}
	\begin{itemize}
		\item<1-> Phone Number Validation\\
			\mintinline{php}|preg_match('/^\+(\d){8,25}$/',Subject); | 
		\item<2-> Password validation\\
			\mintinline{php}{preg_match('/^(?=.*[a-z])(?=.*[A-Z])(?=.*\d)(?=.*[@$!%*?&])[A-Za-z\d@$!%*?&]{8,}$/)',Subject); } 
	\end{itemize}
	
\end{frame}


\end{document}
