\documentclass[aspectratio=169,10pt,t]{beamer}
% \usetheme[
% %%% options passed to the outer theme
% %    progressstyle=fixedCircCnt,   %either fixedCircCnt, movCircCnt, or corner
% %    rotationcw,          % change the rotation direction from counter-clockwise to clockwise
% %    shownavsym          % show the navigation symbols
%   ]{SDUsimple}
\usepackage{SDUtheme/beamerthemeSDUsimple}
% If you want to change the colors of the various elements in the theme, edit and uncomment the following lines
% Change the bar and sidebar colors:
%\setbeamercolor{SDUsimple}{fg=red!20,bg=red}
%\setbeamercolor{sidebar}{bg=red!20}
% Change the color of the structural elements:
%\setbeamercolor{structure}{fg=red}
% Change the frame title text color:
%\setbeamercolor{frametitle}{fg=blue}
% Change the normal text color background:
%\setbeamercolor{normal text}{fg=black,bg=gray!10}
% ... and you can of course change a lot more - see the beamer user manual.
\usepackage{color}
\usepackage{float}
\usepackage{dsfont}                         % Enables double stroke fonts
\usepackage{bm}
\usepackage[utf8]{inputenc}
\usepackage[english]{babel}
\usepackage[T1]{fontenc}
\usepackage{dirtree}
\setlength{\DTbaselineskip}{15pt}
\DTsetlength{0.2em}{1em}{0.2em}{0.4pt}{1.6pt}
\usepackage{booktabs}
\usepackage{minted}
\setminted{linenos=true,breaklines=true,obeytabs=true,tabsize=2}
\usepackage{import}
\newcommand{\includesvg}[2][./]{%
    \immediate\write18{inkscape #1#2.svg --export-latex -D --export-filename=#1#2.pdf }
\import{#1}{#2.pdf_tex}
}

\usepackage{pgf}
\usetikzlibrary{graphs,graphdrawing,quotes}
\usegdlibrary{force,layered,circular}
% Or whatever. Note that the encoding and the font should match. If T1
% does not look nice, try deleting the line with the fontenc.
\usepackage{helvet}
\usefonttheme{professionalfonts}

\newtheorem{algorithm}{Algorithm}
%\newtheorem{problem}{Problem}
\newtheorem{proposition}{Proposition}
% colored hyperlinks
\newcommand{\chref}[2]{%
  \href{#1}{{\usebeamercolor[bg]{SDUsimple}#2}}%
}

\title{API's, REST and SOAP}
\subtitle{ \texttt{localhost:443/api/v1/exam/pass} }
%\date{\today}
\date{ }

\author{
  \textbf{ITI-F20}
}

% - Give the names in the same order as they appear in the paper.
% - Use the \inst{?} command only if the authors have different
%   affiliation. See the beamer manual for an example

\institute[
%  {\includegraphics[scale=0.2]{SDU_segl}}\\ %insert a company, department or university logo
  SDU Robotics\\
  The Maersk Mc-Kinney Moller Institute\\
  University of Southern Denmark
] % optional - is placed in the bottom of the sidebar on every slide
{% is placed on the bottom of the title page
  SDU Robotics\\
  The Maersk Mc-Kinney Moller Institute\\
  University of Southern Denmark

  %there must be an empty line above this line - otherwise some unwanted space is added between the university and the country (I do not know why;( )
}

% specify a logo on the titlepage (you can specify additional logos an include them in
% institute command below
\pgfdeclareimage[height=0.5cm]{titlepagelogo}{SDUgraphics/SDU_logo_new} % placed on the title page
%\pgfdeclareimage[height=1.5cm]{titlepagelogo2}{SDUgraphics/SDU_logo_new} % placed on the title page
\titlegraphic{% is placed on the bottom of the title page
  \pgfuseimage{titlepagelogo}
%  \hspace{1cm}\pgfuseimage{titlepagelogo2}
}

\begin{document}
% the titlepage
{\SDUwavesbg%
\begin{frame}[plain,noframenumbering] % the plain option removes the header from the title page
  \titlepage
\end{frame}}
%%%%%%%%%%%%%%%%

% TOC
\begin{frame}{Agenda}{\vphantom{(y}}
\begin{itemize}
		\item API
		\item REST
		\item Authentication
	\end{itemize}
\end{frame}
%%%%%%%%%%%%%%%


\begin{frame}[fragile]
	\frametitle{API}
	\framesubtitle{200 : \{"succes!"\}}
	\begin{itemize}
		\item Application
		\item Programmer
		\item Interface
	\end{itemize}

	\vfill
	\texttt{GET localhost:8080/xx/mvc/public/api/users}\\
	Response
	\begin{minted}{json}
[
	{
		"user_id":"1","username":"jack"
	},{
		"user_id":"2", "username":"jill"
	}
]	
	\end{minted}
\end{frame}

\begin{frame}[fragile]
	\frametitle{API}
	\framesubtitle{PHP. PHP everywhere}
	\begin{minted}{php}
<?php
public function users () {
	$users = $this->model('User')->getAll();
	echo json_encode($users, JSON_PRETTY_PRINT);
}
	\end{minted}
\end{frame}

\usetikzlibrary{shapes.arrows,shapes.geometric,calc,shadings}
\begin{frame}[fragile]
	\frametitle{REST}
	\framesubtitle{You're still using SOAP? Take a REST}
	\begin{figure}[h]
	\begin{center}
		\begin{tikzpicture}[scale=1, transform shape,every node/.style={color=black}]
		\draw (0,0) node[anchor=east,text width=8cm,minimum height=0.7cm,
		fill=black!50](a){Level 0: The Swamp of POX}
		(0,1) node[anchor=east,text width=7cm,minimum height=0.7cm,
		fill=yellow!30!black](b){Level 1: Resources}
		(0,2) node[anchor=east,text width=6cm,minimum height=0.7cm,
		fill=yellow!45!black](c){Level 2: HTTP Verbs}
		(0,3) node[anchor=east,text width=5cm,minimum height=0.7cm,
		fill=yellow!57!black](d){Level 3: Hypermedia Control}
		($ (a.south east) + (0.5cm,0) $) node [single arrow, minimum height=4.7cm,
		shape border rotate=90,anchor=south,minimum width=1.5cm,
		top color=yellow!60!black, bottom color=black!50!white](e){}
		($ (e.north) + (0,0.4cm) $) node[ellipse,minimum width=1.5cm,
		minimum height=0.6cm,fill=yellow!60!black](g) {}
		(g) node[ellipse,minimum width=1.3cm,
		minimum height=0.45cm,fill=white] {}
		($ (g.north) + (0,0.2cm) $) node{Glory of REST};
	\end{tikzpicture}
	\end{center}
	\end{figure}
\end{frame}

%\begin{frame}[fragile]
%	\frametitle{REST}
%	\framesubtitle{You're still using SOAP? Take a REST}
%	\begin{figure}[h]
%		\centering
%		\def\svgwidth{0.8\textwidth}
%		\scalebox{0.8}{
%			\includesvg[./]{REST}
%		}
%	\end{figure}
%\end{frame}


\begin{frame}[fragile]
	\frametitle{Authentication}
	\framesubtitle{I don't trust you}
	\begin{columns}
		\begin{column}{0.5\textwidth}
			Basic Auth
			\begin{itemize}
				\item Token
				\item Username password
				\item TLS/SSL only
			\end{itemize}
		\end{column}
		\begin{column}{0.5\textwidth}
			oAuth

		\end{column}
	\end{columns}
\end{frame}

\end{document}
